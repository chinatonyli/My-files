\documentclass[oneside]{ctexbook}
\def\pgfsysdriver{pgfsys-dvipdfmx.def}
\usepackage{mathtools}
\usepackage{amssymb}
\usepackage{tikz}
\usepackage{MnSymbol}
\usepackage{indentfirst}
\usepackage{graphicx}
\DeclareGraphicsRule{*}{eps}{*}{}
\usepackage[hmargin={20mm,20mm},vmargin={20mm,10mm}]{geometry}
\let\mathdollar\undefined
\begin{document}

\newtheorem{definition}{定义}[section]
\newtheorem{theorem}{定理}[section]
\newtheorem{lemma}[theorem]{引理}
\newtheorem{corollary}[theorem]{推论}
% \largetriangleup
% \section{. . . }
% \subsection{. . . }
% \subsubsection{. . . }
% \paragraph{. . . }
% \subparagraph{. . . }

\title{我的数学笔记}
\author{李懿峰}
\date{2012年9月}
\maketitle
\tableofcontents
\chapter{全等三角形}

\section{基本概念}
\begin{definition}[全等形]
    \label{qdx}
    完全重合的两个图形叫全等形.  两个图形大小相同,形状相同.
\end{definition}

\begin{definition}[全等三角形]
    满足定义\ref{qdx}的三角形. 符号“$\backcong$”,读作“全等于”. 书写全等式时要把对应定点字母放在对应的位置上.
\end{definition}

\begin{theorem}[全等三角形的性质]
    等全三角形的对应边、对应角、对应中线、对应高、对应角平分线均相等.
\end{theorem}

例:如图\ref{p1},$\triangle ABC\backcong\triangle DEF$
\begin{figure}[htbp]
        \centering
        \begin{tikzpicture}
            \draw
            (5,0) node[below] {$A$}
            --(9,0) node[below] {$C$}
            --(7,3) node[above] {$B$}
            --cycle;
            \draw
            (10,0) node[below] {$D$}
            --(14,0) node[below] {$F$}
            --(12,3) node[above] {$E$}
            --cycle;
        \end{tikzpicture}
        \caption{这两个任意三角形全等}
        \label{p1}
    \end{figure}

\begin{alignat*}{3}
    &\because   \ &\triangle \mathrm{ABC} &\backcong \triangle \mathrm{DEF} &\mbox{(已知)}   \\
    &\therefore \ &          \mathrm{AB}  &=\mathrm{DE}                                               \\
    &           \ &          \mathrm{BC}  &=\mathrm{EF}                                               \\
    &           \ &          \mathrm{AC}  &=\mathrm{DF} &\mbox{(全等三角形的对应边相等)} \\
    &\therefore \ &\angle \mathrm{ABC}    &= \angle \mathrm{DEF} \\
    &           \ &\angle \mathrm{ACB}    &= \angle \mathrm{DFE} \\
    &           \ &\angle \mathrm{BAC}    &= \angle \mathrm{EDF}
\end{alignat*}

\subsection{常见的三角形全等形式}

全等三角形通常会以七种形式出现,如图\ref{p2}所示.

\begin{figure}[htbp]
\centering
\includegraphics{images/1_figs-01.mps}
\includegraphics{images/1_figs-02.mps}
\includegraphics{images/1_figs-04.mps}
\includegraphics{images/1_figs-05.mps}
\includegraphics{images/1_figs-06.mps}
\includegraphics{images/1_figs-07.mps}
\includegraphics{images/1_figs-03.mps}
\caption{七组常见的三角形全等形式}
\label{p2}
\end{figure}

\newpage

\section{全等的判定}

\subsection{一般三角形之判定}
\begin{theorem}[SSS]
    三边对应相等的两三角形全等.
\end{theorem}

\begin{theorem}[SAS]
    两边对应相等且夹角相等的里两三角形全等.
\end{theorem}

\begin{theorem}[ASA]
    两角对应相等且夹边相等的两三角形全等.
\end{theorem}

\begin{theorem}[AAS]
    两个角和其中一个角的对边对应相等的两个三角形全等.
\end{theorem}

\subsection{直角三角形之判定}
\begin{theorem}[HL]
    斜边和一直角边对应相等的两个三角形全等.
\end{theorem}


\end{document}
